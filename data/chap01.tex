% !TeX root = ../sustechthesis-example.tex

\chapter{A}

研究生学位论文撰写,除表达形式上需要符合一定的格式要求外,内容方面上也要遵循一些共性原则。

通常研究生学位论文只能有一个主题(不能是几块工作拼凑在一起),该主题应针对某学科领域中的一个具体问题展开深入、系统的研究,并得出有价值的研究结论。
学位论文的研究主题切忌过大,例如,“中国国有企业改制问题研究”这样的研究主题过大,因为“国企改制”涉及的问题范围太广,很难在一本研究生学位论文中完全研究透彻。

正文主要包括论文的选题背景与目的,论文研究领域发展趋势的分析与述评,论文的选题立论,研究方案阐述,主要研究结果阐释,讨论与研究展望,结论。此部分是论文的主体, 打印时应从另页右页开始,电子版与打印版每一章应另起页。主体部分一般从引言(绪论)开始,以结论结束,分章节论述,层次分明、逻辑性强。

论文正文字数要求为:博士学位论文正文一般为6~10万字(含图表);硕士学位论文正文一般为3~5万字(含图表),如冲突以《南方科技大学研究生学位论文写作指南》为准。

Word count in a doctoral thesis is generally 45 to 50 thousand English words (including tables and figures). And word count in a master's thesis is enerally 22 to 25 thousand English words (including tables and figures). Where any discrepancy arises between this template and the \textit{Guide to thesis writing for graduate students}, the ``Guide'' shall prevail.


\section{论文的语言及表述}

学位论文一般需要用汉语书写。满足下列情况之一, 可以用英文撰写学位论文:

\begin{enumerate}
\item 研究生本身英文水平较高, 且导师为外籍教师或有海外留学或工作经验, 英语水平较高, 可以指导学生用英文撰写论文;
\item 研究生参加双学位国际联合培养项目;
\item 国际研究生。
\end{enumerate}

研究生必须经导师同意后, 方能申请用英文撰写学位论文, 在参加开题评议时需提交中文版开题报告及对应的英文版开题报告, 由开题评议小组对其英文写作水平进行把关, 并给予审核意见。使用英文撰写论文的研究生原则上应采用英文进行答辩。采用英文撰写时须采用中文封面, 且应有800-1000汉字(符)摘要。

中文学位论文,除古汉语研究中涉及的古文字和参考文献中引用的外文文献之外,均采用简体汉字撰写。

研究生学位论文是学术作品,因此其表述要严谨简明,重点突出,专业常识应简写或不写,做到立论正确、数据可靠、说明透彻、推理严谨、文字凝练、层次分明,避免使用文学性质的或带感情色彩的非学术性语言。

论文中如出现一个非通用性的新名词、新术语或新概念,需随即解释清楚。



\section{论文题目的写法}

论文题目应简明扼要地反映论文工作的主要内容,力求精炼、准确,切忌笼统。
论文题目是对研究对象的准确、具体描述,一般要在一定程度上体现研究结论,因此,论文题目不仅应告诉读者这本论文研究了什么问题,更要告诉读者这个研究得出的结论。
例如:“在事实与虚构之间:梅乐、卡彭特、沃尔夫的新闻观”就比“三个美国作家的新闻观研究”更专业、更准确。



\section{摘要的写法}

论文摘要是对论文研究内容的高度概括,应具有独立性和自含性,即应是 一篇简短但意义完整的文章。
通过阅读论文摘要,读者应该能够对论文的研究 方法及结论有一个整体性的了解,因此摘要的写法应力求精确简明。
论文摘要 应包括对问题及研究目的的描述、对使用的方法和研究过程进行的简要介绍、 对研究结论的高度凝练等,重点是结果和结论。

论文摘要切忌写成全文的提纲,尤其要避免“第 1 章……;第 2 章……;……”这样的陈述方式。



\section{引言的写法}

一篇学位论文的引言大致包含如下几个部分:
1、问题的提出;
2、选题背 景及意义;
3、文献综述;
4、研究方法;
5、论文结构安排。
\begin{itemize}
  \item 问题的提出:要清晰地阐述所要研究的问题“是什么”。
    % \footnote{选题时切记要有“问题意识”,不要选不是问题的问题来研究。}
  \item 选题背景及意义:论述清楚为什么选择这个题目来研究,即阐述该研究对学科发展的贡献、对国计民生的理论与现实意义等。
  \item 文献综述:对本研究主题范围内的文献进行详尽的综合述评,“述”的同时一定要有“评”,指出现有研究状态,仍存在哪些尚待解决的问题,讲出自己的研究有哪些探索性内容。
  \item 研究方法:讲清论文所使用的学术研究方法。
  \item 论文结构安排:介绍本论文的写作结构安排。
\end{itemize}



\section{正文的写法}

本部分是论文作者的研究内容,不能将他人研究成果不加区分地掺和进来。
已经在引言的文献综述部分讲过的内容,这里不需要再重复。
各章之间要存在有机联系,符合逻辑顺序。



\section{结论的写法}

结论是对论文主要研究结果、论点的提炼与概括,应精炼、准确、完整,使读者看后能全面了解论文的意义、目的和工作内容。
结论是最终的、总体的结论,不是正文各章小结的简单重复。
结论应包括论文的核心观点,主要阐述作者的创造性工作及所取得的研究成果在本领域中的地位、作用和意义,交代研究工作的局限,提出未来工作的意见或建议。
同时,要严格区分自己取得的成果与指导教师及他人的学术成果。

在评价自己的研究工作成果时,要实事求是,除非有足够的证据表明自己的研究是“首次”、“领先”、“填补空白”的,否则应避免使用这些或类似词语。

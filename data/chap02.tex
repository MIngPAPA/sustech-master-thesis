% !TeX root = ../sustechthesis-example.tex

\chapter{图表示例}

\section{插图}

图片通常在 \env{figure} 环境中使用 \cs{includegraphics} 插入,如图~\ref{fig:example} 的源代码。
建议矢量图片使用 PDF 格式,比如数据可视化的绘图;
照片应使用 JPG 格式;
其他的栅格图应使用无损的 PNG 格式。
注意,LaTeX 不支持 TIFF 格式;EPS 格式已经过时。

\begin{figure}
  \centering
  \includegraphics[width=0.6\linewidth]{example-image-a.pdf}
  \caption{示例图片}
  \label{fig:example}
\end{figure}

\begin{figure}
  \centering
  \includegraphics[width=0.6\linewidth, angle=90]{example-image-a.pdf}
  \caption{示例图片旋转90度}
  \label{fig:exampleRotate90}
\end{figure}

若图或表中有附注,采用英文小写字母顺序编号,附注写在图或表的下方。
% LaTeX 传统上一般将附注的内容同图表的标题写在一起,形成很长的一段文字。

如果一个图由两个或两个以上分图组成时,各分图分别以(a)、(b)、(c)...... 作为图序,并须有分图题。
推荐使用 \pkg{subcaption} 宏包来处理, 比如图~\ref{fig:subfig-a} 和图~\ref{fig:subfig-b}。

\begin{figure}
  \centering
  \subcaptionbox{分图 A\label{fig:subfig-a}}
    {\includegraphics[width=0.45\linewidth]{example-image-a.pdf}}
  \subcaptionbox{分图 B\label{fig:subfig-b}}
    {\includegraphics[width=0.45\linewidth]{example-image-b.pdf}}
  \caption{多个分图的示例}
  \label{fig:multi-image}
\end{figure}



\section{表格}

表应具有自明性。为使表格简洁易读,尽可能采用三线表,如表~\ref{tab:three-line}。
三条线可以使用 \pkg{booktabs} 宏包提供的命令生成。

\begin{table}
  \centering
  \caption{三线表示例}
  \begin{tabular}{ll}
    \toprule
    文件名          & 描述                         \\
    \midrule
    thuthesis.dtx   & 模板的源文件,包括文档和注释 \\
    thuthesis.cls   & 模板文件                     \\
    thuthesis-*.bst & BibTeX 参考文献表样式文件    \\
    thuthesis-*.bbx & BibLaTeX 参考文献表样式文件  \\
    thuthesis-*.cbx & BibLaTeX 引用样式文件        \\
    \bottomrule
  \end{tabular}
  \label{tab:three-line}
\end{table}

表格如果有附注,尤其是需要在表格中进行标注时,可以使用 \pkg{threeparttable} 宏包。使用英文小写字母 a、b、c……顺序编号。

\begin{table}
  \centering
  \begin{threeparttable}[c]
    \caption{带附注以及调整列宽的的表格示例}
    \label{tab:three-part-table}
  \begin{tabular}{C{2cm} L{4cm} R{6cm}}
    \toprule
    2cm          & 4cm & 6cm                         \\
    \midrule
    左右居中的2cm宽度左右居中的2cm宽度\tnote{a}   & 左右居左的4cm宽度左右居左的4cm宽度 & 左右居右的6cm宽度左右居右的6cm宽度\\
    左右居中的2cm宽度左右居中的2cm宽度\tnote{b}   & 左右居左的4cm宽度左右居左的4cm宽度 & 左右居右的6cm宽度左右居右的6cm宽度\\
    \bottomrule
  \end{tabular}
    \begin{tablenotes}
      \item [a] A的注释
      \item [b] B的注释
    \end{tablenotes}
  \end{threeparttable}
\end{table}
如果需要调整表格列宽度, 可以改用命令 \verb|L|, \verb|R|, 或者 \verb|C|, 如 \verb|C{2cm}| 代表居中列宽2cm。

\begin{table}
  \centering
  \caption{合并单元格的三线表}
  \label{tab:merge-cell}
  \begin{tabular}{lcccc}
  \toprule
    Metaclass & \multicolumn{2}{c}{A-B} & \multicolumn{2}{c}{C-D} \\ \cmidrule(l){1-1} \cmidrule(lr){2-3} \cmidrule(r){4-5}
    Class & A & B & C & D \\ \midrule
    L1 & 1 & 2 & 3 & 4 \\
    L2 & 1 & 2 & 3 & 4 \\
  \bottomrule
  \end{tabular}
\end{table}

如有辅助线要求可以使用 \verb|\cmidrule| 命令。在连续使用时,可以使用一组圆括号括起来的参数 \verb|l|、\verb|r|或\verb|l<距离>|、\verb|r<距离>|表示间距的表格线可以在左右向内缩短一小段,表\ref{tab:merge-cell}展示了效果。

表格如果想要与页面等宽,可以使用 \pkg{tabularx} 宏包,如表格\ref{tab:textwith-example}所示。
模版定义了一些扩展命令,实现一些排版需求。\verb|X| 两端对齐, \verb|Y| 左对齐, \verb|Z| 右对齐,或者 \verb|A| 居中对齐。

\begin{table}
  \centering
  \caption{同页宽的表格实例}
  \label{tab:textwith-example}
  \begin{tabularx}{\textwidth}{YXAZ}
    \toprule
    Cell with text aligned to the left & 1 & 2 & 3\\ \midrule
    4 & Cell with justified text & 5 & 6\\ \midrule
    7 & 8 & Cell with centered text & 9\\ \midrule
    10 & 11 & 12 & Cell with text aligned to the right \\
    \bottomrule
  \end{tabularx}
\end{table}



如果您要排版的表格长度超过一页,那么推荐使用 \pkg{longtable} 或者 \pkg{supertabular}
宏包,模板对 \pkg{longtable} 进行了相应的设置,所以用起来可能简单一些。
表~\ref{tab:performance} 就是 \pkg{longtable} 的简单示例。

\begin{longtable}[c]{ccccclr}
  \caption{实验数据(超长表格示例)}\label{tab:performance}\\
  \toprule[1.5pt]
   测试程序 & \multicolumn{1}{c}{正常运行} & \multicolumn{1}{c}{同步} & \multicolumn{1}{c}{检查点} & \multicolumn{1}{c}{卷回恢复}
  & \multicolumn{1}{c}{进程迁移} & \multicolumn{1}{c}{检查点} \\
  & \multicolumn{1}{c}{时间 (s)}& \multicolumn{1}{c}{时间 (s)}&
  \multicolumn{1}{c}{时间 (s)}& \multicolumn{1}{c}{时间 (s)}& \multicolumn{1}{c}{
    时间 (s)}&  文件(KB)\\\midrule[1pt]
  \endfirsthead
    \multicolumn{7}{c}{\continuetable 实验数据(超长表格示例)}\\
  \toprule[1.5pt]
   测试程序 & \multicolumn{1}{c}{正常运行} & \multicolumn{1}{c}{同步} & \multicolumn{1}{c}{检查点} & \multicolumn{1}{c}{卷回恢复}
  & \multicolumn{1}{c}{进程迁移} & \multicolumn{1}{c}{检查点} \\
  & \multicolumn{1}{c}{时间 (s)}& \multicolumn{1}{c}{时间 (s)}&
  \multicolumn{1}{c}{时间 (s)}& \multicolumn{1}{c}{时间 (s)}& \multicolumn{1}{c}{
    时间 (s)}&  文件(KB)\\\midrule[1pt]
  \endhead
  \bottomrule[1.5pt]
  \multicolumn{7}{r}{续下页}
  \endfoot
  \endlastfoot
  CG.A.2 & 23.05 & 0.002 & 0.116 & 0.035 & 0.589 & 32491 \\
  CG.A.4 & 15.06 & 0.003 & 0.067 & 0.021 & 0.351 & 18211 \\
  CG.A.8 & 13.38 & 0.004 & 0.072 & 0.023 & 0.210 & 9890 \\
  CG.B.2 & 867.45 & 0.002 & 0.864 & 0.232 & 3.256 & 228562 \\
  CG.B.4 & 501.61 & 0.003 & 0.438 & 0.136 & 2.075 & 123862 \\
  CG.B.8 & 384.65 & 0.004 & 0.457 & 0.108 & 1.235 & 63777 \\
  MG.A.2 & 112.27 & 0.002 & 0.846 & 0.237 & 3.930 & 236473 \\
  MG.A.4 & 59.84 & 0.003 & 0.442 & 0.128 & 2.070 & 123875 \\
  MG.A.8 & 31.38 & 0.003 & 0.476 & 0.114 & 1.041 & 60627 \\
  MG.B.2 & 526.28 & 0.002 & 0.821 & 0.238 & 4.176 & 236635 \\
  MG.B.4 & 280.11 & 0.003 & 0.432 & 0.130 & 1.706 & 123793 \\
  MG.B.8 & 148.29 & 0.003 & 0.442 & 0.116 & 0.893 & 60600 \\
  LU.A.2 & 2116.54 & 0.002 & 0.110 & 0.030 & 0.532 & 28754 \\
  LU.A.4 & 1102.50 & 0.002 & 0.069 & 0.017 & 0.255 & 14915 \\
  LU.A.8 & 574.47 & 0.003 & 0.067 & 0.016 & 0.192 & 8655 \\
  LU.B.2 & 9712.87 & 0.002 & 0.357 & 0.104 & 1.734 & 101975 \\
  LU.B.4 & 4757.80 & 0.003 & 0.190 & 0.056 & 0.808 & 53522 \\
  LU.B.8 & 2444.05 & 0.004 & 0.222 & 0.057 & 0.548 & 30134 \\
  EP.A.2 & 123.81 & 0.002 & 0.010 & 0.003 & 0.074 & 1834 \\
  EP.A.4 & 61.92 & 0.003 & 0.011 & 0.004 & 0.073 & 1743 \\
  EP.A.8 & 31.06 & 0.004 & 0.017 & 0.005 & 0.073 & 1661 \\
  EP.B.2 & 495.49 & 0.001 & 0.009 & 0.003 & 0.196 & 2011 \\
  EP.B.4 & 247.69 & 0.002 & 0.012 & 0.004 & 0.122 & 1663 \\
  EP.B.8 & 126.74 & 0.003 & 0.017 & 0.005 & 0.083 & 1656 \\
  EP.A.2 & 123.81 & 0.002 & 0.010 & 0.003 & 0.074 & 1834 \\
  \bottomrule[1.5pt]
\end{longtable}


\section{源代码}

% 推荐使用 \pkg{listing} 环境嵌入 \pkg{minted} 环境高亮代码。\verb|linenos| 参数控制代码行号显示。\pkg{minted} 环境需要 Python 环境编译,并安装 Pygement 包,否则会编译失败。
% 引用效果如代码 \ref{lst:sample-code-minted}。

% 也可以
使用 \pkg{listings} 环境高亮代码。参数较为复杂,请自行搜索或查阅文档。引用效果如代码 \ref{lst:sample-code-listings}。示例使用 \pkg{minipage} 环境嵌套一层的原因是防止换页中被插入其他浮动体,结合实际情况,按需使用 \pkg{minipage},例如如需要跨页代码就无需使用 \pkg{minipage}。
% 但是,\textbf{不建议}混用 \pkg{listings} 环境和 \pkg{minted} 环境,会导致如上编号重复的错误,二选一即可。

% \begin{listing}[!ht]
% \caption{C++ 代码示例(使用 \pkg{minted} 高亮)}
% \label{lst:sample-code-minted}
% \begin{minted}[xleftmargin=20pt,linenos]{cpp}
% #include <cstdio>
% #include <cstdlib>
% #include <iostream>
% using namespace std;
% unsigned short i;
% int main() {
%   for (i = 0; i <= 5; i++) {
%     // whatever
%   }
%   return 0;
% }
% \end{minted}
% \end{listing}

\noindent% 取消 minipage 的缩进
\begin{minipage}{\linewidth}
\begin{lstlisting}[language=java,caption={Java 代码示例(使用 \pkg{listings} 高亮)},xleftmargin=20pt,label={lst:sample-code-listings}]
class HelloWorldApp {
    public static void main(String[] args) {
        System.out.println("Hello World!"); // Display the string.
        for (int i = 0; i < 100; ++i) {
            System.out.println(i);
        }
    }
}
\end{lstlisting}
\end{minipage}

\section{伪代码}


推荐使用 \pkg{algorithm2e} 宏包中的 \pkg{algorithm} 环境书写伪代码。\pkg{algorithm2e} 可选参数 \verb|linesnumbered| 控制代码行号显示。引用效果如算法 \ref{algo:sample-pseudocode}。

\begin{algorithm}
  \caption{Simulation-optimization heuristic}
  \label{algo:sample-pseudocode}
  \KwData{current period $t$, initial inventory $I_{t-1}$, initial capital $B_{t-1}$, demand samples}
  \KwResult{Optimal order quantity $Q^{\ast}_{t}$}
  $r\leftarrow t$\;
  $\Delta B^{\ast}\leftarrow -\infty$\;
  \While{$\Delta B\leq \Delta B^{\ast}$ and $r\leq T$}{$Q\leftarrow\arg\max_{Q\geq 0}\Delta B^{Q}_{t,r}(I_{t-1},B_{t-1})$\;
  $\Delta B\leftarrow \Delta B^{Q}_{t,r}(I_{t-1},B_{t-1})/(r-t+1)$\;
  \If{$\Delta B\geq \Delta B^{\ast}$}{$Q^{\ast}\leftarrow Q$\;
  $\Delta B^{\ast}\leftarrow \Delta B$\;}
  $r\leftarrow r+1$\;}
\end{algorithm}


\begin{algorithm}
  \caption{SumExample}
  \label{algo:sample-pseudocode2}
  \KwResult{$s$}
  $s\leftarrow 0$ \Comment*[r]{这是默认多行注释}
  \Comment{这是默认独占一行的注释}
  \SetNoFillComment %
  \Comment{这是在取消独占一行后的注释}
  \SetFillComment
  \Comment{这是恢复独占一行的注释}
  % \LeftComment{左对齐}
  \ForEach{$i\in [1, 100]$}{
    \uIf{$i\% 3= 0$}{
      $s\leftarrow s+i$   \SingleComment*[r]{这是单行注释,一个没有end的if}
    }\uElseIf{$i\%3=1$}{
      \Break  \TriComment*{这是三角形的单行注释,一个没有end的else if}
    }\Else{
      \Continue \Comment*[r]{这是超长多行注释,关于伪代码的if-then-else详细查看https://texdoc.org/serve/algorithm2e/0的10.4}
    }
  }
  \Return{s} \;
\end{algorithm}

\newpage
\section{测试}

图题在图之下,段前空 6 磅,段后空 12 磅。图整体前后距离未定义,目前默认距离:段前空 12 磅,段后空 12 磅。

图前,图前,图前,图前,图前,图前,图前,图前,图前,图前,图前。

\begin{figure}[htbp]
  \centering
  \includegraphics[height=12bp,width=50bp]{example-image-a.pdf}\hspace*{6bp}
  \includegraphics[height=6bp,width=50bp]{example-image-a.pdf}
  \caption{图高度为12bp vs 6bp}
\end{figure}

图后,图后,图后,图后,图后,图后,图后,图后,图后,图后,图后。

表题在表之上,段前空 12 磅,段后空 6 磅。表整体前后距离未定义,目前默认距离:段前空 12 磅,段后空 12 磅。

表前,表前,表前,表前,表前,表前,表前,表前,表前,表前,表前。

\begin{table}[htbp]
  \centering
  \caption{简单表格}
    \begin{tabular}{cc}
    \toprule
    column1 & column2\\
    \midrule
    column1 & column2\\
    \bottomrule
    \end{tabular}
  \label{label}
\end{table}

表后,表后,表后,表后,表后,表后,表后,表后,表后,表后,表后。
\emph{图表前后是否有空行不影响图表与正文之间的距离}。

% !TeX root = ../sustechthesis-example.tex

\chapter{引用文献的标注}

模板支持 BibTeX 和 BibLaTeX 两种方式处理参考文献。
下文主要介绍 BibTeX 配合 \pkg{natbib} 宏包的主要使用方法。


\section{顺序编码制}

\thusetup{
  cite-style = super,
}
% \citestyle{super} % 全局上标数字模式
% \bibliographystyle{sustechthesis-numeric}

在顺序编码制下,默认的 \cs{cite} 命令同 \cs{citep} 一样,即序号置于方括号中,引文页码会放在括号外。统一处引用的连续序号会自动用短横线连接。如多次引用同一文献,可能需要标注页码,例如:引用第二页\cite[2]{zhangkun1994},引用第五页\cite[5]{zhangkun1994}。

\begin{tabular}{l@{\quad$\Rightarrow$\quad}l}
  \verb|\cite{zhangkun1994}|               & \cite{zhangkun1994}   {\kaishu 不带页码的上标引用}            \\
  \verb|\citet{zhangkun1994}|              & \citet{zhangkun1994}              \\
  \verb|\citep{zhangkun1994}|              & \citep{zhangkun1994}              \\
  \verb|\cite[42]{zhangkun1994}|           & \cite[42]{zhangkun1994} {\kaishu 手动带页码的上标引用}          \\
  \verb|\cite{zhangkun1994,zhukezhen1973}| & \cite{zhangkun1994,zhukezhen1973}  {\kaishu 一次多篇文献的上标引用}  \\
\end{tabular}


注意,引文参考文献的每条都要在正文中标注
\cite{zhangkun1994,zhukezhen1973,dupont1974bone,zhengkaiqing1987,%
  jiangxizhou1980,jianduju1994,merkt1995rotational,mellinger1996laser,%
  bixon1996dynamics,mahui1995,carlson1981two,taylor1983scanning,%
  taylor1981study,shimizu1983laser,atkinson1982experimental,%
  kusch1975perturbations,guangxi1993,huosini1989guwu,wangfuzhi1865songlun,%
  zhaoyaodong1998xinshidai,biaozhunhua2002tushu,chubanzhuanye2004,%
  who1970factors,peebles2001probability,baishunong1998zhiwu,%
  weinstein1974pathogenic,hanjiren1985lun,dizhi1936dizhi,%
  tushuguan1957tushuguanxue,aaas1883science,fugang2000fengsha,%
  xiaoyu2001chubanye,oclc2000about,scitor2000project%
}。
引用测试:2个连续引用\cite{zhangkun1994,zhukezhen1973},2个间隔\cite{zhangkun1994,dupont1974bone},3个连续\cite{zhangkun1994,zhukezhen1973,dupont1974bone}。

如参考文献中需要使用上标或者下标,使用数学环境书写 \verb|$\mathrm{Ba}_{3}\mathrm{CoSb}_{2}\mathrm{O}_{9}$|,例如该文献\cite{kamiya2018nature}。根据 \pkg{gbt7714} 规定著者姓名自动转为大写。西文的题名、期刊名的大小写不自动处理,需要自行处理以符合信息资源本身文种的习惯用法。

\section{著者-出版年制}

著者-出版年制下的 \cs{cite} 跟 \cs{citet} 一样。

\thusetup{
  cite-style = author-year,
}
\begin{tabular}{l@{\quad$\Rightarrow$\quad}l}
  \verb|\cite{zhangkun1994}|                & \cite{zhangkun1994}                \\
  \verb|\citet{zhangkun1994}|               & \citet{zhangkun1994}               \\
  \verb|\citep{zhangkun1994}|               & \citep{zhangkun1994}               \\
  \verb|\cite[42]{zhangkun1994}|            & \cite[42]{zhangkun1994}            \\
  \verb|\citep{zhangkun1994,zhukezhen1973}| & \citep{zhangkun1994,zhukezhen1973} \\
\end{tabular}

\vskip 2ex
% \thusetup{
%   cite-style = super,
% }
注意,引文参考文献的每条都要在正文中标注
\cite{zhangkun1994,zhukezhen1973,dupont1974bone,zhengkaiqing1987,%
  jiangxizhou1980,jianduju1994,merkt1995rotational,mellinger1996laser,%
  bixon1996dynamics,mahui1995,carlson1981two,taylor1983scanning,%
  taylor1981study,shimizu1983laser,atkinson1982experimental,%
  kusch1975perturbations,guangxi1993,huosini1989guwu,wangfuzhi1865songlun,%
  zhaoyaodong1998xinshidai,biaozhunhua2002tushu,chubanzhuanye2004,%
  who1970factors,peebles2001probability,baishunong1998zhiwu,%
  weinstein1974pathogenic,hanjiren1985lun,dizhi1936dizhi,%
  tushuguan1957tushuguanxue,aaas1883science,fugang2000fengsha,%
  xiaoyu2001chubanye,oclc2000about,scitor2000project%
}。


\section{条目编写}

\subsection{支持三级目录显示}

支持三级目录显示


\subsection{条目要求}

条目要求首行左缩进 2 个汉字符,避免悬挂缩进。如需使用带括号的条目列表,请自行添加 \verb|label=<style>| 参数。
下面是两个例子,还有更多用法,查阅 \pkg{enumitem} 宏包的文档。

默认条目序号:

\verb|\begin{enumerate} ... \end{enumerate}|

\begin{enumerate}
  \item 一级
  \begin{enumerate}
    \item 二级
    \begin{enumerate}
      \item 三级
      \begin{enumerate}
        \item 四级,《写作要求》未定义,请自行定义或者选择。
      \end{enumerate}
    \end{enumerate}
  \end{enumerate}
\end{enumerate}

自定义序号样式定义如表~\ref{tab:enum-style}。

\begin{table}[h]
  \centering
  \caption{条目样式选项}
  \label{tab:enum-style}
  \begin{tabular}{@{}ll@{}}
  \toprule
  \textbf{Code}          & \textbf{Description}                      \\ \midrule
  \textbackslash{}alph   & Lowercase letter (a, b, c, ...)           \\
  \textbackslash{}Alph   & Uppercase letter (A, B, C, ...)           \\
  \textbackslash{}arabic & Arabic number (1, 2, 3, ...)              \\
  \textbackslash{}roman  & Lowercase Roman numeral (i, ii, iii, ...) \\
  \textbackslash{}Roman  & Uppercase Roman numeral (I, II, III, ...) \\ \bottomrule
  \end{tabular}
\end{table}

\subsubsection{条目测试}


条目前文字,条目前文字,条目前文字,条目前文字,条目前文字,条目前文字,条目前文字,条目前文字,条目前文字,条目前文字,条目前文字。

\begin{enumerate}
  \item 一级条目,超长行。南方科技大学,南方科技大学,南方科技大学,南方科技大学,南方科技大学,南方科技大学,南方科技大学,南方科技大学,南方科技大学,南方科技大学,南方科技大学,南方科技大学,南方科技大学。
  \item 一级条目,南方科技大学,南方科技大学,南方科技大学。
  \item 一级条目,南方科技大学,南方科技大学,南方科技大学。
  \begin{enumerate}
    \item 二级条目,超长行。南方科技大学,南方科技大学,南方科技大学,南方科技大学,南方科技大学,南方科技大学,南方科技大学,南方科技大学,南方科技大学,南方科技大学,南方科技大学,南方科技大学,南方科技大学。
    \item 二级条目,南方科技大学,南方科技大学,南方科技大学。
    \item 二级条目,南方科技大学,南方科技大学,南方科技大学。
    \item 二级条目,南方科技大学,南方科技大学,南方科技大学。
    \item 二级条目,南方科技大学,南方科技大学,南方科技大学。
    \item 二级条目,南方科技大学,南方科技大学,南方科技大学。
    \begin{enumerate}
      \item 三级条目,南方科技大学,南方科技大学,南方科技大学。
      \item 三级条目,南方科技大学,南方科技大学,南方科技大学。
      \item 三级条目,南方科技大学,南方科技大学,南方科技大学。
      \begin{enumerate}
        \item \textbf{四级条目及之后的条目无规定序号格式,请自行设定选择。}
      \end{enumerate}
    \end{enumerate}
  \end{enumerate}
\end{enumerate}

条目后文字,条目后文字,条目后文字,条目后文字,条目后文字,条目后文字,条目后文字,条目后文字,条目后文字,条目后文字,条目后文字。
